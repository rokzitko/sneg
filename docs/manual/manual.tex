\documentclass[a4paper,10pt,openany]{book}
\usepackage{a4wide}
\usepackage{times}
\usepackage{amsmath, amssymb, graphicx}

\usepackage[section]{placeins}
\usepackage{booktabs}


\begin{document}

\title{SNEG\\
{\it Mathematica package for calculations with non-commuting
operators of the second quantization algebra}\\
MANUAL
}

\author{Copyright (C) 2006 Rok Zitko, rok.zitko@ijs.si}
\date{}

\maketitle


\newcommand{\nc}{\mathrm{nc}}
\newcommand{\vc}[1]{\mathbf{#1}}
\newcommand{\ket}[1]{| #1 \rangle}
\newcommand{\bra}[1]{\langle #1 |}

\section{Introduction}

The purpose of this manual is to describe the notations and
conventions used in the SNEG library. The manual is rather abstract
and concise, it focuses more on the design of the library than on
real-world examples; it is not a tutorial. The examples in directory
{\tt examples/} should thus be studied in conjunction with reading
this text. For details, one should consult the library source code,
which is the only true reference.

\section{Non-commuting multiplication}

The cornerstone of SNEG is the non-commuting multiplication function {\tt
nc}. {\it Mathematica}'s built-in multiplication function {\tt Times} has
{\tt Orderless} attribute, which corresponds to the mathematical property of
commutativity, therefore it is unsuitable. Another built-in function, {\tt
NonCommutativeMultiply}, has {\tt Flat} attribute, which corresponds to the
mathematical property of associativity. Non-commuting multiplication should
indeed be associative; unfortunately, {\tt Flat} attribute also affects
pattern matching, which proved to be undesirable for our purposes due to
decreased computational performance.

An operator string is a sequence of second-quantization operators,
such as $c^\dag_{k\uparrow} c^\dag_{k\downarrow} c_{k\downarrow}
c_{k\uparrow}$.  In SNEG, it would be represented as {\tt
  nc[c[CR,k,UP], c[CR,k,DO], c[AN,k,DO], c[AN,k,UP]]} (see
Section~\ref{ops} on operators).

Function {\tt nc} has the following basic properties:

\begin{itemize}
\item Associativity: for example, $\nc[a, \nc[b, c]] = \nc[a, b, c]$,
  where $a$, $b$ and $c$ are operators.

\item Linearity: for example, $\nc[\alpha a+\beta b, c] = \alpha
  \nc[a,c] + \beta \nc[b,c]$, where $\alpha$ and $\beta$ are numbers,
  and $a$, $b$ and $c$ are operators.  Linearity implies
  distributivity. Furthermore, numerical objects are factored out from
  operator strings.

\item Product of zero terms equals one, $\nc[]=1$.

\item Product of one term equals the term itself, $\nc[a]=a$.
\end{itemize}

In addition, fermionic operators are automatically normal ordered by
anti-commuting the creation operators to the left and annihilation
operators to the right using the default (canonical) or user-defined
anti-commutation relations (see Section~\ref{acmt} on anti-commutation
relations and operator ordering). This produces equivalent results for
equivalent strings of operators and enables automatic expression
simplifications. This property is the {\tt nc} equivalent of argument
sorting in the commuting multiplication function {\tt Times}.

Dirac bras and kets (see Section~\ref{dirac} on Dirac notation) can
also appear in {\tt nc} multiplication expressions. They are
automatically shifted to the right-most side of operator expressions.
The default behavior is that bras and kets commute with operators,
i.e. it is assumed that they belong to a different Hilbert space than
the one where the second quantization operators operate.


\section{Operators}
\label{ops}

In SNEG, operators are indexed objects, for example {\tt c[t, sigma]}.
The expression head, {\tt c}, is the operator name, while {\tt t,
  sigma} are indeces which indicate, for example, if the operator is a
creation or an annihilation operator, and the associated degrees of
freedom such as spin. The operator character of {\tt c} must be
declared using the {\tt snegfermionoperators} function: for example {\tt
  snegfermionoprators[c]}.

By convention, the first index indicates if the operator creates or
annihilates a particle. Two constants are predefined for this purpose:
{\tt CR=0} and {\tt AN=1}. Another convention is that the spin index
in the case of $S=1/2$ operators is on the last position. For convenience,
two further constants are predefined: {\tt UP=1} and {\tt DO=0}.

For a single orbital, the creation operators are thus {\tt c[CR, UP]}
and {\tt c[CR, DO]}, while the corresponding annihilation operators
are {\tt c[AN, UP]} and {\tt c[AN, DO]}. (Conjugation can be performed
using the function {\tt conj}.)


\section{Numbers and numeric expressions}

SNEG must be able to differentiate operators from numbers and other
numeric expressions in order to factor out non-operator parts of
operator expressions. Using {\tt snegrealconstants[x1, x2,...]}, {\tt
  x1}, {\tt x},\ldots are defined to be real constants (invariant
under conjugation). Using {\tt snegcomplexconstants[z1, z2,...]}, {\tt
  z1}, {\tt z2},\ldots are defined to be complex constants.  Finally,
using {\tt snegfreeindexes[k, sigma,...]} we notify SNEG that {\tt k},
{\tt sigma},\ldots are indexes: this is required to factor out numerical
expressions such as {\tt KroneckerDelta[k1,k2]} out of operator strings.


\section{Anti-commutation relations and operator ordering}
\label{acmt}

Two properties of operators must be known to SNEG to perform automatic
reorderings and simplifications. These are anti-commutation relations
and the vacuum state. The first affects how the transpositions of
operators are performed, while the latter determines the conventional
ordering of operators in operator strings. 

The default anti-commutation relation for fermionic operators are the
usual cannonical anti-commutation relations,
$\{c^\dag_\alpha,c_\beta\} = \delta_{\alpha\beta}$,
$\{c^\dag_\alpha,c^\dag_\beta\} = 0$, and $\{c_\alpha,c_\beta\} = 0$.
Operators with different heads simply anti-commute. The value of the
$\{c^\dag_\alpha,c_\beta\}$ anti-commutator is defined by the funciton
{\tt acmt[c, \{indexes1\}, \{indexes2\}}. The default behaviour is to
literally compare indexes1 and indexes2: 1 is returned if they are equal,
and 0 otherwise. 

{\tt ordering[c]=SEA}, {\tt ordering[c]=EMPTY}.

\section{Expression-building functions}

SNEG includes several functions that can be used to build operator
expressions such as occupation number, spin operator, spin-spin scalar
product, charge-charge repulsion, etc.

Unless otherwise specified, parameters to such functions are operators with
their type ({\tt CR} or {\tt AN}) and spin indeces removed, for example {\tt
c[]}. To build the occupancy (number) operator for orbital $c$, we therefore
call {\tt number[c[]]}, which returns an expression equivalent to
$\sum_\sigma c^\dag_\sigma c_\sigma$. The sum over spin is automatically
performed. Additional indexes are given as arguments to {\tt c[]}, for
example {\tt number[c[k]]} gives an expression corresponding to $\sum_\sigma
c^dag_{k\sigma} c_{k\sigma}$, i.e. the occupancy operator for a state with
wavenumber $k$.

Functions in this group are:
%
\begin{itemize}
\item {\tt number[c[], sigma]} - number of particles with spin {\tt sigma}
in orbital $c$, i.e. $n_\sigma=c^\dag_\sigma c_\sigma$.

\item {\tt number[c[]]} - charge density (number of particles),
i.e. $n=n_\uparrow + n_\downarrow$.

\item {\tt spinx[c[]]}, {\tt spiny[c[]]}, {\tt spinz[c[]]} - spin density,
i.e. $\vc{S} = \sum_{\alpha\beta} c^\dag_\alpha (1/2 \boldsymbol{\sigma})
c_\beta$, where $\alpha$ and $\beta$ are spin indexes and
$\boldsymbol{\sigma}$ is the vector of Pauli matrices.

\item {\tt spinplus[c[]]} and {\tt spinminus[c[]]} - spin raising and
spin lowering operators, $S^+=S_x+I S_y$ and $S^-=S_x-I S_y$.

\item {\tt spinss[c[]]} - the total spin operator squared, $\vc{S}^2$.

\item {\tt spinspin[c[1], c[2]]} - scalar product of two spin operators
for different (or equal) orbitals, $\vc{S}_1 \cdot \vc{S}_2$.
Transverse and longitudinal parts of the scalar product are obtained
using {\tt spinspinxy[c[1], c[2]]} and {\tt spinspinz[c[1], c[2]]}.

\item {\tt hubbard[c[]]} - Hubbard's local electron-electron repulstion
operator, $n_\uparrow n_\downarrow$.

\item {\tt nambu[c[]]} - Nambu spinor, $\eta = \{ c^\dag_\uparrow, (-1)^n
c_\downarrow \}$.

\item {\tt isospin[c[]]} - isospin operator, $\vc{I} = \sum_{\alpha\beta}
\eta_\alpha (1/2 \boldsymbol{\sigma}) \eta_\beta$, where $\eta$ is the Nambu
spinor, $\alpha$ and $\beta$ are isospin indexes and $\boldsymbol{\sigma}$
is the vector of Pauli matrices.

\item {\tt hop[c[1], c[2]]} - electron hopping operator
$\sum_\sigma (c^\dag_{1\sigma} c_{2\sigma} + c^\dag_{2\sigma} c_{1\sigma})$.

\item {\tt twohop[c[1], c[2]]} - two-electron hopping operator
$c^\dag_{1\downarrow} c^\dag_{1\uparrow} c_{2\downarrow} c_{2\uparrow}
+\text{H.c.}$.

\item {\tt projection[c[], pr]} - projection operator.
{\tt pr = PROJ0 | PROJUP | PROJDO | PROJ2 | PROJ1 | PROJ02}; {\tt PROJ0}
projects to zero-occupancy states, $(1-n_\uparrow)(1-n_\downarrow)$, {\tt
PROJUP} to spin-up states, $n_\uparrow(1-n_\downarrow)$, {\tt PROJDO} to
spin-down states, $n_\downarrow(1-n_\uparrow)$, {\tt PROJ2} to
double-occupancy states, $n_\uparrow n_\downarrow$, {\tt PROJ1} to
single-occupancy states, and {\tt PROJ02} to zero- and double-occupancy
states.

\end{itemize}


\section{Expression manipulation functions}

Function {\tt invertspin} inverts the spins of all operators appearing
in an expression. The convention that the spin index appears at the
last position must be followed.

Function {\tt conj} calculates the Hermitian conjugate of an expression.

Function {\tt vev} calculates the vacuum expectation value of an operator
expression; {\tt vevwick} does the same using Wick's theorem.

Function {\tt wick} writes an operator string using normal ordered
strings of operators and contractions (Wick's theorem).

{\tt komutator[a,b]} calculates the commutator $[a,b]$, while
{\tt antikomutator[a,b]} calculates the the anti-commutator $\{a,b\}$.


\section{Dirac's bra and ket notation}
\label{dirac}

Dirac's bras $\langle i,j,\ldots|$ and kets $|i,j,\rangle$ are
represented by expressions {\tt bra[i,j,...]}  and {\tt ket[i,j,...]},
where {\tt i}, {\tt j},\ldots denote the quantum numbers that
determine the corresponding state.

By default, the values of brakets are determined using the Kronecker's
delta, i.e. $\langle i_1, j_1,\ldots | i_2, j_2, \ldots =
\delta_{i_1i_2} \delta_{j_1j_2} \ldots$.

Bras and kets may appear in {\tt nc} strings. They do not commute,
i.e. the braket $\langle i|j \rangle$ is different from the projector
$|j\rangle \langle i|$.

Different quantum numbers correspond to different argument positions.
{\tt Null} placeholders can be used in place of unspecified quantum
numbers. Neighboring bras and kets are ``concatenated'' if they have
compatible patterns of {\tt Null} placeholders. For example: {\tt
  nc[ket[i, Null], ket[Null,j]]=ket[i,j]}. From the mathematical point
of view, this corresponds to a direct (tensor) product of states.


\section{{\tt VACUUM} vector}

Keyword {\tt VACUUM} corresponds to a vacuum state. By default, if an
annihilation operator is applied to {\tt VACUUM} from the left, the
result is 0. Correspondingly, if a creation operator is applied to
{\tt conj[VACUUM]} from the right, the result is 0.

By definition, {\tt nc[conj[VACUUM], VACUUM]=1}.


\section{State vectors: operator and occupation number representations}

A state can be represented either as the effect of string of creation
operators on a (unspecified) vacuum state, or as a set of occupation
numbers. If the vacuum is known, a one-to-one mapping between the two
representations can be established.

In the creation operator representation, a state is just an operator
expression. For example, a one-particle state $1/\sqrt{2}
(c^\dag_{1\uparrow} + c^\dag_{2\uparrow}) \ket{0}$ would be described as
{\tt 1/Sqrt[2] (c[CR,1,UP]+c[CR,2,UP])}, a two-particle state
$c^\dag_{1\uparrow} c^\dag_{1\downarrow}$ as {\tt nc[c[CR,1,UP],
c[CR,1,DO]]}, etc.

In the occupation number representation, a state is given by a vector of
occupation numbers, {\tt vc[0,1,0,0,1,0,\dots]}. Each position in the vector
corresponds to some orbital with a given spin. 

A mapping between operators and positions in the vector is established using
{\tt makebasis} function. For two orbitals, for example, the mapping is
defined by {\tt makebasis[c[1], c[2]]}. Position 1 then corresponds to
$c^\dag_{1\uparrow}$, position 2 to $c^\dag_{1\downarrow}$, position 3 to
$c^\dag_{2\uparrow}$, and position 4 to $c^\dag_{2\downarrow}$. The mapping
can be accessed as the global variable {\tt BASIS}, or using {\tt op2ndx}
and {\tt ndx2op} mapping functions. 

The zero-particle vacuum can be obtained using {\tt vacuum}. Operator
expressions can be applied to occupation number vectors using {\tt ap}. One
can go from creation operator representation to occupation number
representation and back using {\tt ops2vc} and {\tt vc2ops} mapping
functions.

Scalar products of vectors are given by {\tt scalarproductop[a,b]} and {\tt
scalarproductvc[a,b]}. Norms can be calculated using {\tt normop[a]} and
{\tt normvc[a]}.

An operator given by a second-quantisation operator expression {\tt O} can
be transformed using {\tt matrixrepresentationop[O, l]} and {\tt
matrixrepresentationvc[O, l]} to its matrix representation in a subspace
spanned by states given in a list {\tt l}.

A vector can be decomposed in a given subspace using {\tt decomposeop[v, l]}
and {\tt decomposevc[v, l]}.

A set of vectors can be orthogonalized in a given subspace using
{\tt orthogop[vs, l]} and {\tt orthogvc[vs, l]}, where {\tt vs} are
the vectors and {\tt l} is a list of vectors spanning the subspace.


\section{Basis sets}

In SNEG, basis sets are lists of subspace basis sets. A subspace basis
set consists of a list of invariant quantum numbers that characterize
the invariant subspace and a list of states in either creation operator
or occupation number representation.

Function {\tt qszbasis[l]} constructs a basis set with subspaces with
well-defined charge, $Q$, and spin projection, $S_z$, quantum numbers. List
{\tt l} is the list of operators that defined the orbitals; it is usually
the same list as the one passed as the argument to {\tt makebasis}.

Function {\tt qsbasis[l]} constructs a basis set with subspaces with
well-defined charge, $Q$, and total spin, $S$, quantum numbers.

Conversions of basis sets from creation operator to occupation
number representation can be performed using {\tt bzvc2bzop} 
and {\tt bzop2bzvc} functions.


\section{Symbolic sums}

{\tt sum[expr, \{k, sigma,{\ldots}\}]}

\end{document}
